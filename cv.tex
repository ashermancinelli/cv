\documentclass{mycv}

\name{Asher Mancinelli}
\address{Pacific Northwest National Laboratory \\ 902 Battelle Boulevard \\ Richland, WA  99352  USA}
\email{asher.mancinelli@pnnl.gov}
\github{ashermancinelli}
\linkedin{asher-mancinelli-bb4a56144}

\begin{document}

\maketitle

\section{Research \\ Interests}

I am interested in low-level code optimization, concurrency, distributed systems, and systems programming.
My current focuses include:

\begin{itemize}
  \item Smooth and sustainable ports of CPU-bound HPC codes to GPUs.
  \item Software development best practices.
  \item Concurrent and distributed design patterns.
  \item Performance tuning.
\end{itemize}

\section{Professional \\ Experience}

\subsection{Pacific Northwest National Laboratory}[Richland, WA USA]
\begin{positions}
  \entry{ExaSGD Software Development Thrust Lead}{April 2020~--~Present}
  \entry{HPC Software Developer}{June 2019~--~March 2020}
  \entry{Machine Learning/HPC Software Developer}{June 2018~--~June 2019}
\end{positions}

\begin{itemize}
  \item Led development team to port high-performance linear solver \textit{HiOp} to run on GPUs from several several vendors (AMD, NVIDIA).
  \item Led several team-wide hackathons of up to ~40 members of 5 national laboratories to port power-grid simulations to GPU.
  \item Reimplemented and refactored parts of NIH protein-folding codebase with 30,000+ users worldwide (can be found \href{https://github.com/Electrostatics/apbs}{\underline{on its github repo}})
  \item Increased performance of a biological simulation by ~$2,500\%$ (from 2 weeks to 8 minutes).
  \item Increased performance of an LED simulation by ~$15,000\%$ (from 6 weeks to 4 minutes).
  \item Optimization work described as \textit{the posterchild for how PNNL should operate} by a panel of external collaborators.
  \item Reproduced closed-source research with a seq-2-seq neural network for OCR (\textit{Rosetta} network).
  \item Developed new implementation for an \verb|MPI_Gatherv| call using tree-based reduction algorithm.
  \item Led training sessions on code optimization and HPC techniques at PNNL.
  \item Gave talks about optimizing research code and developing inter-language interfaces to be performant.
\end{itemize}

\subsection{Micron Technologies}[Washington D.C. Metro Area]
\begin{positions}
  \entry{Data Science \& Analytics Intern}{June 2017~--~August 2017}
\end{positions}

\begin{itemize}
  \item Invented algorithm to cluster wafer defects increasing company profit by significant margin.
  \item Wrote statistical regression model to filter large amounts of financial data which increased bargaining power of procurement department.
  \item Developed time-series model to predict server failure that decreased server downtime.
\end{itemize}

\section{Education}

\subsection{Eastern Washington University}[Cheney, WA]
\vspace{-\parskip}%
\begin{itemize}[label={}]
  \item Bachelors of Science in Computer Science, minor in Mathematics \printdate{September 2016~--~March 2020}
  \item Graduated with \textit{magna cum laude}
  \item Member of Dean's List
  \item Received three scholarships for exceptional performance in the Math dept.
  \item Placed $1^{st}$ in programming competition \textit{PineCodes}. \printdate{2019}
\end{itemize}

\section{Skills}

\begin{description}
  \item[Programming] C/C++, CUDA, CMake, MPI, Python, \LaTeX, Shell Scripting
  \item[Tools] Git, Linux, Vim/Tmux, GNU/Lua module systems
  \item[Soft Skills] Public speaking, technical writing, editorial review
\end{description}

\section{Misc}

\begin{itemize}
  \item Gave talk at \href{https://fullstacktc.org/user/ashermancinelli}{ \underline{ local developer meetup } }. \printdate{2017}
  \item Member of technical editorial review board for Brett Slatkin's \textit{Effective Python}. \printdate{2019}
  \item Contributor to various open-source projects, such as \textit{PyTorch}, \textit{ElasticSearch}, and \textit{Hiop} (see GitHub profile).
  \item Reader of ISO C++ TS papers discussing the development of the language.
\end{itemize}

\end{document}
